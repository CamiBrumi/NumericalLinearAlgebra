\documentclass{article}
\usepackage[utf8]{inputenc}

\title{Pràctica 3 ALN: Mínims quadrats}
\author{Camelia Brumar & Leonor Lamsdorff-Galagane}
\date{08 May 2017}

\usepackage{natbib}
\usepackage{graphicx}

\begin{document}

\maketitle

\section{La demostració de que $p_{M}(x)$ és el polinomi “òptim”}
Demostració que la projecció ortogonal és mínima (parèntesi en text)
$$ \left(parentesi en una equacio \right) $$
$$ \frac{numerador}{denominador} $$
$$ \pi \tau \sum_1^n$$


\section{Justificació del mètode}
A continuació justificarem el mètode utilitzat per trobar les solucions de les equacions normals i l'explicarem breument.

En aquesta pràctica hem utilitzat els mètodes d'ortogonalització, que aconsegueixen la reducció a forma triangular mitjançant l'ús de matrius ortogonals. Aquests mètodes estan basats en la descomposició $QR$ de matrius. Com en els mètodes gaussians, s'observen dues parts en el procés d'ortogonalització:

\begin{itemize}
  \item En la primera, es factoritza A en la forma:
  $$ A =  QR, $$
  on $Q$ és ortogonal i $R$ triangular superior.
  \item En la segona, s'acaba la resolució del sistema $Ax = QRx = b$, resolent el sistema triangular superior $Rx = Q^\top b$ pel mètode de substitució cap enrere. Arribem a aquest sistema senzill seguint els següents passos: 
  $$A^\top Ax = A^\top b \;,$$
  com que $A = QR$
  $$\left(QR \right)^\top QRx = \left(QR \right)^\top b$$
  $$R^\top Q^\top  QRx = R^\top Q^\top  b$$
  sabent que $Q$ és una matriu ortogonal, i.e, $Q^\top =\ Q^{-1}$, obtenim
  $$R^\top Rx = R^\top Q^\top  b$$
  i finalment multipliquem per la inversa de $R^\top$ a les dues bandes de la igualtat, aconseguint el sistema
  $$Rx = Q^\top b \; .$$
\end{itemize}

Aquesta factorització l'hem realitzat amb el mètode d'ortogonalització mo\-di\-fi\-cat de Gram-Schmidt.\newline

\newline
\textbf{Mètode d'ortogonalització modificat de Gram-Schmidt} \newline
Començant amb $A_{1} = A$, una vegada coneguda
$$A_{k} = \left(q_{1}\: \ldots{}\: q_{k-1}\: a_{k}^{(k)}\: \ldots{} \:a_{n}^{(k)} \right)$$

amb columnes $q_{j}\;\left(j = 1 \div k - 1  \right)$ i $a_{s}^{(k)}\, \left(s = k \div n \right)$ complint amb les relacions d'ortogonalitat
$$q_{j}^\top q_{l} = \delta_{jl},\; q_{j}^\top a_{s} = 0\, \left(j, l = 1 \div k-1,\, s = k \div n \right),$$
es normalitza la $k$-èssima columna i s'ortogonalitzen, respecte a ella, totes les que la segueixen:
$$r_{kk} = \left| \left|\; a_{k}^{(k)}\; \right| \right|_{2}, \; q_{k} = \frac{a_{k}^{(k)}}{r_{kk}}\;; $$ 
$$r_{ks} = q_{k}^\top a_{s}^{(k)}\; , \; \; a_{s}^{(k+1)} = a_{s}^{(k)} - r_{ks}q_{k} \; \; \left( s = k+1 \div n \right) .$$
S'obté
$$A_{k+1} = \left( q_{1} \; \ldots{} \; q_{k} \; a_{k+1}^{(k+1)} \; \ldots{} \; a_{n}^{(k+1)} \right) \; ,$$

verificant les mateixes condicions d'ortogonalitat anteriors, substituint $k$ per $k+1 \; \left( k = 1 \div n \right).$ \newline
Després de $n$ passos,
$$A_{n+1} = \left(q_{1} \; q_{2} \ldots{} q_{n} \right)$$
és una matriu ortogonal perquè $q_{j}^{T}q_{l} = \delta_{jl} \; \left( j, l = 1 \div n \right).$ \newline
\newline
Les matrius $Q = A_{n+1}$ i $R = r_{(ks)}$ formen una factorització $QR$ de $A$.
\end{document}
